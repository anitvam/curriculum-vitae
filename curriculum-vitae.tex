\documentclass{resume}

\begin{document}

\fontfamily{ppl}\selectfont

\noindent
\begin{tabularx}{\linewidth}{@{}m{0.8\textwidth} m{0.2\textwidth}@{}}
{
    \Large{Martina Baiardi} \newline
    \small{
        \clink{
            \href{mailto:martina.baiardi.98@gmail.com}{martina.baiardi.98@gmail.com} 
        } \newline
        Cesena, Italy
    }
} & 
{
    \hfill
}
\end{tabularx}
\begin{center}
\begin{tabularx}{\linewidth}{@{}*{2}{X}@{}}
% left side %
{
    \csection{EDUCATION}{\small
        \begin{itemize}
            \item \frcontent{M.Sc. Engineering and Computer Science}{\textit{University of Bologna - Cesena}}{}{From 2020 onwards}
            \item \frcontent{B.Sc. Engineering and Computer Science}{\textit{University of Bologna - Cesena} \newline
            \textbf{Thesis}: Controllo e Scalabilità Automatizzati in Cluster Kubernetes \newline
            \textbf{Supervisor: Vittorio Ghini}}{110L/110}{From 2017 to 2020}
            \item \frcontent{Acounting degree in Business Information Systems}{\textit{I.T.E. R.Serra - Cesena}}{100/100}{2017}
        \end{itemize}
    }
    \csection{EXPERIENCE}{\small
    \begin{itemize}
        \item \frcontent{Tutor for the course 'Sistemi Virtualizzati'}{Laurea in Tecnologie dei Sistemi Informatici}{University of Bologna - Cesena \newline \textbf{Supervisor: Vittorio Ghini}}{From March 2022 onwards}
        \item \frcontent{Tutor for the course 'Metodi Numerici'}{Laurea in Ingegneria e Scienze Informatiche}{University of Bologna - Cesena \newline \textbf{Supervisor: Lucia Romani}}{From March 2022 onwards}
        \item \frcontent{Web Developer at \href{https://www.blockvision.it/}{\underline{Blockvision}}}{}{}{From September 2020 onwards}
        \item \frcontent{Curricular Internship}{University of Bologna - Cesena, Italy}{Study of a Kubernetes cluster: main aspects, tools and realization of a cluster on virtual machines}{From March to May 2020}
    \end{itemize}
    }
} 
% end left side %
& 
% right side %
{
    \csection{SKILLS}{\small
        \begin{itemize}
            \item \textbf{Languages} \newline
            {\footnotesize Java, Scala, Javascript, Vue, PHP, C, C++, C\#, Python, Kotlin}{}{}
            \item \textbf{Technologies} \newline
            {\footnotesize Docker, Kubernetes, Ansible}{}{}
            \item \textbf{Patterns \& Practices} \newline
            {\footnotesize Object Oriented Programming, Functional  Programming}
            \item \textbf{Project Management} \newline
            {\footnotesize Scrum, Agile}
        \end{itemize}
    }
    \csection{PROJECTS}{\small
        \begin{itemize}
            \item \frcontent{Bunny Survival \clink{\href{https://github.com/anitvam/pps-bunny}{[pps-bunny]}}}{\textit{August to October 2021}}{An open-source simulator of a bunny population made for academic purposes under the course "Paradigmi di Programmazione e Sviluppo" inspired from the interactive simulator \href{https://phet.colorado.edu/en/simulations/natural-selection}{\underline{Natural Selection}} developed by University of Colorado}{Scala}
            \item \frcontent{Ca' Foscari Jisho}{\textit{May to July 2021}}{Web application for italian-japanese translations made for academic purposes under the course "Applicazioni and Servizi Web" in collaboration with the research group \href{http://www.edrdg.org}{\underline{EDRDG}} from Ca' Foscari University of Venice}{Vue, Ansible, Javascript}
        \end{itemize}
    }
    \csection{INTERESTS}{\small
        \begin{itemize}
            \setlength\itemsep{-5px}
            \item {\footnotesize Kubernetes}
            \item {\footnotesize Continuous Integration \& Continuous Deployment}
            \item {\footnotesize Microservices}
            \item {\footnotesize User Experience and User Interface}
            \item {\footnotesize Design}
        \end{itemize}
    }
}
\end{tabularx}
\end{center}
\end{document}
