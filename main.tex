\documentclass[10pt,a4paper]{altacv}
\usepackage{cv-style}
\begin{document}
    \name{Martina Baiardi}
    \tagline{Research Fellow}
    \personalinfo{%
        % You can add your own with \printinfo{symbol}{detail}
        \email{martina.baiardi.98@gmail.com}
    %   \phone{+49-173-6895039}
        \location{Cesena, Italy}
        \github{github.com/anitvam}
        \linkedin{linkedin.com/in/martina-baiardi}
        %\orcid{orcid.org/0000-0000-0000-0000} % Obviously making this up too. If you want to use this field (and also other academicons symbols), add "academicons" option to \documentclass{altacv}
    }

    %% Make the header extend all the way to the right, if you want. Extend the right margin by 8cm (=6.8cm marginparwidth + 1.2cm marginparsep)
    \begin{adjustwidth}{}{-8cm}
        \makecvheader
    \end{adjustwidth}

\begin{adjustwidth}{}{-9cm}
    \cvsection{Education}

    \cvblock{M.Sc. in Engineering and Computer Science}{University of Bologna}{2020--2023}{Cesena, Italy}
    \textbf{110L/110} \newline
    \textbf{Thesis}: JACOP: Programming BDI Agents with Pluggable Concurrency Model \newline
    \textbf{Supervisor: Giovanni Ciatto} \newline \textbf{Co-Supervisors: Andrea Omicini, Jomi F. Hübner}

    \divider

    \cvblock{Curricular Internship}{University of Bologna}{03--05/2020}{Cesena, Italy}
    Study of a Kubernetes cluster: main aspects, tools and realization of a cluster on virtual machines

    \divider

    \cvblock{B.Sc. in Engineering and Computer Science}{University of Bologna}{2017--2020}{Cesena, Italy}
    \textbf{110L/110} \newline
    \textbf{Thesis}: Kubernetes cluster autoscaling and management: a case study\newline
    \textbf{Supervisor: Vittorio Ghini}

    \divider

    \cvblock{Accounting degree in Business Information Systems}{ITE "R. Serra"}{2017}{Cesena, Italy}
    \textbf{100/100}

    \divider

\end{adjustwidth}

\begin{adjustwidth}{}{-9cm}
    \cvsection{Research Contracts}

    \cvblock{Research Scholarship for hardware system management within the StairwAI project}{DISI - University of Bologna}{01--07/2023}{Cesena, Italy \bigskip}
    \textbf{Supervisors: Andrea Borghesi, Giovanni Ciatto}

    \divider

\end{adjustwidth}

\begin{adjustwidth}{}{-9cm}
    \cvsection{Teaching}

    \cvblock{Tutor for the course "Programmazione ad Oggetti"}{DISI - Laurea in Ingegneria e Scienze Informatiche}{A.Y. 2022/2023}{Cesena, Italy}
    \textbf{Supervisor: Mirko Viroli}

    \divider

    \cvblock{Tutor for the course "Sistemi Virtualizzati"}{DISI - Laurea in Tecnologie dei Sistemi Informatici}{A.Y. 2021/2022, \newline A.Y. 2022/2023}{Cesena, Italy}
    \textbf{Supervisor: Vittorio Ghini}

    \divider

    \cvblock{Tutor for the course "Metodi Numerici"}{DISI - Laurea in Ingegneria e Scienze Informatiche}{A.Y. 2021/2022}{Cesena, Italy}
    \textbf{Supervisor: Lucia Romani}

    \divider

\end{adjustwidth}
\clearpage

\begin{adjustwidth}{}{-9cm}
    \cvsection{Previous work experience}

    \cvblock{Web Developer}{Blockvision}{2020--2022}{Cesena, Italy}
    \worknote

    \divider

    \cvblock{Waitress}{Il Localino}{2019--2021}{Gambettola, Italy}
    \worknote

    \divider

    \cvblock{Bartender}{Rockin Park}{Summer \newline 2017--2019}{Cesena, Italy}
    \worknote

    \divider

    \cvblock{Wardrobe Attendant}{Vidia Club}{Winter \newline 2016--2019}{Cesena, Italy}
    \worknote

    \divider
    
\end{adjustwidth}


\begin{adjustwidth}{}{-9cm}
    \cvsection{Professional Skills}

    \cvskillblock{Programming Languages}{Kotlin, Java, Scala, Javascript, C, Bash, Python, PHP, Sql, Prolog, C\#, C++}
    \divider

    \cvskillblock{Other Languages}{Markdown, YAML, \LaTeX{}, HTML, CSS, XML, JSON}
    \divider

    \cvskillblock{Technologies}{Git, Docker, Kubernetes, Ansible, Android, Vue, Spring, Jacamo, Jason}
    \divider

    \cvskillblock{Programming Paradigms}{Object Oriented Programming, Functional Programming, Agent Oriented Programming, Logic Programming, Event Driven Programming, Imperative/Procedural Programming}
    \divider

    \cvskillblock{Project Management}{Scrum, Agile}
    \divider
\end{adjustwidth}

\begin{adjustwidth}{}{-9cm}
    \cvsection{Portfolio}

    \textbf{JaKtA}\\
    \github{\href{https://github.com/jakta-bdi}{github.com/jakta-bdi}}\\
    \textit{10/2022--03/2023}\\ \smallskip
    \underline{Ja}son-like \underline{K}o\underline{t}lin \underline{A}gents (JaKtA): a Kotlin internal DSL meant to seamlessly integrate BDI agents into a mainstream programming language, thus adding AOP to Kotlin as an additional paradigm, retaining its toolchain, libraries, and OOP/FP abstractions. The project was made for academic purposes for the ``Intelligent System Engineering'' course and then extended for the Master Degree Thesis.\\ \smallskip
    {\small \notesymbol \hspace{0.5em} Paper submitted for publication to PAAMS 2023 conference}\\
    \smallskip
    \textbf{Language:} Kotlin\\
    \textbf{Keywords:} Domain Specific Language, Gradle Multi-Project, Domain Driven Design

    \divider

    \textbf{Smart Waste Collection}\\
    \github{\href{https://github.com/SmartWasteCollection}{github.com/SmartWasteCollection}}\\
    \book{\href{https://smartwastecollection.github.io/documentation/}{smartwastecollection.github.io/documentation}}\\
    \textit{04--10/2022}\\ \smallskip
    The aim of the project was to optimize the management of waste collection using \textbf{Digital Twins} over a simulated environment.
    The project was made for academic purposes for ``Laboratorio dei Sistemi Software'' and ``Pervasive Computing'' courses.\\
    \smallskip

    \textbf{Language:} Kotlin\\
    \textbf{Keywords:} Digital Twins, Microservices, Azure Cloud, CI/CD, Domain Driven Design

    \divider

    \clearpage

    \textbf{Smart Subway}\\
    \textit{01--02/2022}\\ \smallskip
    The objective was to create a smart system for the management and monitoring of metropolitan stations.
    The topic was used for the development of two projects for two different Master Degree courses: the first was ``Smart City e Tecnologie Mobili'' course regarding an implementation using AWS technologies,
    the second was ``Project Management'' course to carry out all Project Management Lifecycle phases.\\ \smallskip
    \textbf{Keywords:} AWS Cloud, IoT, RFID

    \divider

    \textbf{Bunny Survival}\\
    \github{\href{https://github.com/anitvam/pps-bunny}{github.com/anitvam/pps-bunny}}\\
    \textit{08--10/2021}\\ \smallskip
    An open-source simulator of a bunny population made for academic purposes under the course ``Paradigmi di Programmazione e Svilupp'' inspired from the interactive simulator \href{https://phet.colorado.edu/en/simulations/natural-selection}{\underline{Natural Selection}} developed by University of Colorado.\\ \smallskip
    \textbf{Language:} Scala\\
    \textbf{Keywords:} Scrum

    \divider

    \textbf{\textbf{Ca' Foscari Jisho}}\\
    \link{\href{https://jisho.unive.it/}{jisho.unive.it}}\\
    \textit{05--07/2021} \\ \smallskip
    Web application for italian-japanese translations made for academic purposes under the course ``Applicazioni and Servizi Web'' in collaboration with the research group \href{http://www.edrdg.org}{\underline{\textbf{EDRDG}}} of \textbf{Ca' Foscari University} in Venice, Italy.\\ \smallskip
    \textbf{Keywords:} Vue, Docker, Ansible, Javascript, Agile

    \divider
\end{adjustwidth}

\begin{adjustwidth}{}{-9cm}
    \cvsection{INTERESTS}

    \cvtag{Kubernetes}
    \cvtag{CI \& CD}
    \cvtag{Microservices}
    \cvtag{User Experience and User Interface}
    \cvtag{Design}
    \cvtag{Photography}

    \divider
\end{adjustwidth}

\end{document}
