%%%%%%%%%%%%%%%%%
% This is an example CV created using altacv.cls (v1.1, 21 November 2016) written by
% LianTze Lim (liantze@gmail.com), based on the
% Cv created by BusinessInsider at http://www.businessinsider.my/a-sample-resume-for-marissa-mayer-2016-7/?r=US&IR=T
%
%% It may be distributed and/or modified under the
%% conditions of the LaTeX Project Public License, either version 1.3
%% of this license or (at your option) any later version.
%% The latest version of this license is in
%%    http://www.latex-project.org/lppl.txt
%% and version 1.3 or later is part of all distributions of LaTeX
%% version 2003/12/01 or later.
%%%%%%%%%%%%%%%%

%% If you want to use \orcid or the
%% academicons icons, add "academicons"
%% to the \documentclass options.
%% Then compile with XeLaTeX or LuaLaTeX.
% \documentclass[10pt,a4paper,academicons]{altacv}
\documentclass[10pt,a4paper]{altacv}

%% AltaCV uses the fontawesome and academicon fonts
%% and packages.
%% See texdoc.net/pkg/fontawecome and http://texdoc.net/pkg/academicons for full list of symbols.
%% When using the "academicons" option,
%% Compile with LuaLaTeX for best results. If you
%% want to use XeLaTeX, you may need to install
%% Academicons.ttf in your operating system's font %% folder.


% Change the page layout if you need to
\geometry{left=1cm,right=9cm,marginparwidth=6.8cm,marginparsep=1.2cm,top=1cm,bottom=1cm}

% Change the font if you want to.

% If using pdflatex:
\usepackage[utf8]{inputenc}
\usepackage[T1]{fontenc}
\usepackage[default]{lato}
\usepackage{hyperref}

% If using xelatex or lualatex:
% \setmainfont{Lato}

% Change the colours if you want to
\definecolor{Primary}{HTML}{32746D}
\definecolor{SlateGrey}{HTML}{2E2E2E}
\definecolor{LightGrey}{HTML}{666666}
\colorlet{heading}{Primary}
\colorlet{accent}{Primary}
\colorlet{emphasis}{SlateGrey}
\colorlet{body}{LightGrey}

% Change the bullets for itemize and rating marker
% for \cvskill if you want to
\renewcommand{\itemmarker}{{\small\textbullet}}
\renewcommand{\ratingmarker}{\faCircle}

\begin{document}
    \name{Martina Baiardi}
    \tagline{Computer Science and Engineering Student}
    \personalinfo{%
    % Not all of these are required!
    % You can add your own with \printinfo{symbol}{detail}
        \email{martina.baiardi.98@gmail.com}
%   \phone{+49-173-6895039}
        \location{Cesena, Italy}
        \github{github.com/anitvam}
        \linkedin{linkedin.com/in/martina-baiardi/}

    % I'm just making this up though.
%   \orcid{orcid.org/0000-0000-0000-0000} % Obviously making this up too. If you want to use this field (and also other academicons symbols), add "academicons" option to \documentclass{altacv}
    }

%% Make the header extend all the way to the right, if you want. Extend the right margin by 8cm (=6.8cm marginparwidth + 1.2cm marginparsep)
    \begin{adjustwidth}{}{-8cm}
        \makecvheader
    \end{adjustwidth}

%% Provide the file name containing the sidebar contents as an optional parameter to \cvsection.
%% You can always just use \marginpar{...} if you do
%% not need to align the top of the contents to any
%% \cvsection title in the "main" bar.
    \cvsection[pagesidebar]{Education}

    \cvevent{M.Sc. in Engineering and Computer Science}{University of Bologna}{2020 -- ongoing}{Cesena, Italy}

    \divider

    \cvevent{B.Sc. in Engineering and Computer Science}{University of Bologna}{2017 -- 2020}{Cesena, Italy}
    \textbf{110L/110} \newline
    \textbf{Thesis}: Controllo e Scalabilità Automatizzati in Cluster Kubernetes \newline
    \textbf{Supervisor: Vittorio Ghini}

    \divider

    \cvevent{Accounting degree in Business Information Systems}{Istituto Tecnico Economico "R. Serra"}{2017}{Cesena, Italy}
    \textbf{100/100}
    \divider

    \cvsection{Experience}
    \cvevent{Tutor for the course "Sistemi Virtualizzati"}{Laurea in Tecnologie dei Sistemi Informatici}{March 2022 to September 2022}{University of Bologna}
    \textbf{Supervisor: Vittorio Ghini}

    \divider

    \cvevent{Tutor for the course "Metodi Numerici"}{Laurea in Ingegneria e Scienze Informatiche}{March 2022 to September 2022}{University of Bologna}
    \textbf{Supervisor: Lucia Romani}

    \divider

    \cvevent{Web Developer}{\href{https://www.blockvision.it/}{\underline{Blockvision}}}{September 2020 -- ongoing}{Cesena, Italy}
    \divider

    \cvevent{Curricular Internship}{University of Bologna - Cesena, Italy}{March to May 2020}{Cesena, Italy}
    \textbf{Study of a Kubernetes cluster: main aspects, tools and realization of a cluster on virtual machines}

    \clearpage

\end{document}
